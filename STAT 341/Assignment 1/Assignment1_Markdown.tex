% Options for packages loaded elsewhere
\PassOptionsToPackage{unicode}{hyperref}
\PassOptionsToPackage{hyphens}{url}
%
\documentclass[
]{article}
\usepackage{amsmath,amssymb}
\usepackage{lmodern}
\usepackage{ifxetex,ifluatex}
\ifnum 0\ifxetex 1\fi\ifluatex 1\fi=0 % if pdftex
  \usepackage[T1]{fontenc}
  \usepackage[utf8]{inputenc}
  \usepackage{textcomp} % provide euro and other symbols
\else % if luatex or xetex
  \usepackage{unicode-math}
  \defaultfontfeatures{Scale=MatchLowercase}
  \defaultfontfeatures[\rmfamily]{Ligatures=TeX,Scale=1}
\fi
% Use upquote if available, for straight quotes in verbatim environments
\IfFileExists{upquote.sty}{\usepackage{upquote}}{}
\IfFileExists{microtype.sty}{% use microtype if available
  \usepackage[]{microtype}
  \UseMicrotypeSet[protrusion]{basicmath} % disable protrusion for tt fonts
}{}
\makeatletter
\@ifundefined{KOMAClassName}{% if non-KOMA class
  \IfFileExists{parskip.sty}{%
    \usepackage{parskip}
  }{% else
    \setlength{\parindent}{0pt}
    \setlength{\parskip}{6pt plus 2pt minus 1pt}}
}{% if KOMA class
  \KOMAoptions{parskip=half}}
\makeatother
\usepackage{xcolor}
\IfFileExists{xurl.sty}{\usepackage{xurl}}{} % add URL line breaks if available
\IfFileExists{bookmark.sty}{\usepackage{bookmark}}{\usepackage{hyperref}}
\hypersetup{
  pdftitle={Assignment 1},
  pdfauthor={Fenglin Chen},
  hidelinks,
  pdfcreator={LaTeX via pandoc}}
\urlstyle{same} % disable monospaced font for URLs
\usepackage[margin=1in]{geometry}
\usepackage{color}
\usepackage{fancyvrb}
\newcommand{\VerbBar}{|}
\newcommand{\VERB}{\Verb[commandchars=\\\{\}]}
\DefineVerbatimEnvironment{Highlighting}{Verbatim}{commandchars=\\\{\}}
% Add ',fontsize=\small' for more characters per line
\usepackage{framed}
\definecolor{shadecolor}{RGB}{248,248,248}
\newenvironment{Shaded}{\begin{snugshade}}{\end{snugshade}}
\newcommand{\AlertTok}[1]{\textcolor[rgb]{0.94,0.16,0.16}{#1}}
\newcommand{\AnnotationTok}[1]{\textcolor[rgb]{0.56,0.35,0.01}{\textbf{\textit{#1}}}}
\newcommand{\AttributeTok}[1]{\textcolor[rgb]{0.77,0.63,0.00}{#1}}
\newcommand{\BaseNTok}[1]{\textcolor[rgb]{0.00,0.00,0.81}{#1}}
\newcommand{\BuiltInTok}[1]{#1}
\newcommand{\CharTok}[1]{\textcolor[rgb]{0.31,0.60,0.02}{#1}}
\newcommand{\CommentTok}[1]{\textcolor[rgb]{0.56,0.35,0.01}{\textit{#1}}}
\newcommand{\CommentVarTok}[1]{\textcolor[rgb]{0.56,0.35,0.01}{\textbf{\textit{#1}}}}
\newcommand{\ConstantTok}[1]{\textcolor[rgb]{0.00,0.00,0.00}{#1}}
\newcommand{\ControlFlowTok}[1]{\textcolor[rgb]{0.13,0.29,0.53}{\textbf{#1}}}
\newcommand{\DataTypeTok}[1]{\textcolor[rgb]{0.13,0.29,0.53}{#1}}
\newcommand{\DecValTok}[1]{\textcolor[rgb]{0.00,0.00,0.81}{#1}}
\newcommand{\DocumentationTok}[1]{\textcolor[rgb]{0.56,0.35,0.01}{\textbf{\textit{#1}}}}
\newcommand{\ErrorTok}[1]{\textcolor[rgb]{0.64,0.00,0.00}{\textbf{#1}}}
\newcommand{\ExtensionTok}[1]{#1}
\newcommand{\FloatTok}[1]{\textcolor[rgb]{0.00,0.00,0.81}{#1}}
\newcommand{\FunctionTok}[1]{\textcolor[rgb]{0.00,0.00,0.00}{#1}}
\newcommand{\ImportTok}[1]{#1}
\newcommand{\InformationTok}[1]{\textcolor[rgb]{0.56,0.35,0.01}{\textbf{\textit{#1}}}}
\newcommand{\KeywordTok}[1]{\textcolor[rgb]{0.13,0.29,0.53}{\textbf{#1}}}
\newcommand{\NormalTok}[1]{#1}
\newcommand{\OperatorTok}[1]{\textcolor[rgb]{0.81,0.36,0.00}{\textbf{#1}}}
\newcommand{\OtherTok}[1]{\textcolor[rgb]{0.56,0.35,0.01}{#1}}
\newcommand{\PreprocessorTok}[1]{\textcolor[rgb]{0.56,0.35,0.01}{\textit{#1}}}
\newcommand{\RegionMarkerTok}[1]{#1}
\newcommand{\SpecialCharTok}[1]{\textcolor[rgb]{0.00,0.00,0.00}{#1}}
\newcommand{\SpecialStringTok}[1]{\textcolor[rgb]{0.31,0.60,0.02}{#1}}
\newcommand{\StringTok}[1]{\textcolor[rgb]{0.31,0.60,0.02}{#1}}
\newcommand{\VariableTok}[1]{\textcolor[rgb]{0.00,0.00,0.00}{#1}}
\newcommand{\VerbatimStringTok}[1]{\textcolor[rgb]{0.31,0.60,0.02}{#1}}
\newcommand{\WarningTok}[1]{\textcolor[rgb]{0.56,0.35,0.01}{\textbf{\textit{#1}}}}
\usepackage{longtable,booktabs,array}
\usepackage{calc} % for calculating minipage widths
% Correct order of tables after \paragraph or \subparagraph
\usepackage{etoolbox}
\makeatletter
\patchcmd\longtable{\par}{\if@noskipsec\mbox{}\fi\par}{}{}
\makeatother
% Allow footnotes in longtable head/foot
\IfFileExists{footnotehyper.sty}{\usepackage{footnotehyper}}{\usepackage{footnote}}
\makesavenoteenv{longtable}
\usepackage{graphicx}
\makeatletter
\def\maxwidth{\ifdim\Gin@nat@width>\linewidth\linewidth\else\Gin@nat@width\fi}
\def\maxheight{\ifdim\Gin@nat@height>\textheight\textheight\else\Gin@nat@height\fi}
\makeatother
% Scale images if necessary, so that they will not overflow the page
% margins by default, and it is still possible to overwrite the defaults
% using explicit options in \includegraphics[width, height, ...]{}
\setkeys{Gin}{width=\maxwidth,height=\maxheight,keepaspectratio}
% Set default figure placement to htbp
\makeatletter
\def\fps@figure{htbp}
\makeatother
\setlength{\emergencystretch}{3em} % prevent overfull lines
\providecommand{\tightlist}{%
  \setlength{\itemsep}{0pt}\setlength{\parskip}{0pt}}
\setcounter{secnumdepth}{-\maxdimen} % remove section numbering
\ifluatex
  \usepackage{selnolig}  % disable illegal ligatures
\fi

\title{Assignment 1}
\author{Fenglin Chen}
\date{21/09/2021}

\begin{document}
\maketitle

\hypertarget{question-1-a}{%
\subsubsection{Question 1 a)}\label{question-1-a}}

\(\alpha(\mathcal{P})\) is location invariant if
\(\alpha(\mathcal{P}+b) = \alpha(\mathcal{P})\).

First, substitute \(\sigma^4\) into the equation. \[
\alpha(\mathcal{P}) = \frac{\frac{1}{N} \sum_{u \in \mathcal{P}}\left( y_u -\overline{y} \right)^4}{ [\frac{1}{N}\sum_{u \in \mathcal{P}}(y_u-\overline y)^2]^2 }-3
\]

Then, add a constant \(b\) to all elements. Since the population mean is
location equivariant, it also gains \(b\). \$\$ \begin{align*}
\begin{split}
\alpha(\mathcal{P}+b) &= \frac{\frac{1}{N} \sum_{u \in \mathcal{P}}\left( y_u + b -(\overline{y} + b) \right)^4}{ [\frac{1}{N}\sum_{u \in \mathcal{P}}(y_u + b-(\overline y + b))^2]^2 }-3\\

&= \frac{\frac{1}{N} \sum_{u \in \mathcal{P}}\left( y_u + b -\overline{y} - b \right)^4}{ [\frac{1}{N}\sum_{u \in \mathcal{P}}(y_u + b-\overline y - b)^2]^2 }-3\\

&= \frac{\frac{1}{N} \sum_{u \in \mathcal{P}}\left( y_u -\overline{y} \right)^4}{ [\frac{1}{N}\sum_{u \in \mathcal{P}}(y_u -\overline y)^2]^2 }-3 \\

&= \alpha(\mathcal{P})\\

\end{split}
\end{align*} \$\$ Therefore, excess kurtosis is location invariant.

\newpage

\hypertarget{question-1-b}{%
\subsubsection{Question 1 b)}\label{question-1-b}}

\(\alpha(\mathcal{P})\) is scale invariant if
\(\alpha(m*\mathcal{P}) = \alpha(\mathcal{P})\).

Similar to part a), begin by substituting \(\sigma^4\) into the
equation. \[
\alpha(\mathcal{P}) = \frac{\frac{1}{N} \sum_{u \in \mathcal{P}}\left( y_u -\overline{y} \right)^4}{ [\frac{1}{N}\sum_{u \in \mathcal{P}}(y_u-\overline y)^2]^2 }-3
\] Then, multiply all elements by \(m\). Since the population mean is
scale equivariant, it is also multiplied by \(m\). \[
\alpha(m*\mathcal{P}) = \frac{\frac{1}{N} \sum_{u \in \mathcal{P}}\left( m*y_u - m*\overline{y} \right)^4}{ [\frac{1}{N}\sum_{u \in \mathcal{P}}(m*y_u-m*\overline y)^2]^2 }-3
\] Factor out the multiplier \(m\) in both the numerator and
denominator. \[
\alpha(m*\mathcal{P}) = \frac{\frac{1}{N} \sum_{u \in \mathcal{P}}(m\left(y_u - \overline{y} \right))^4}{ [\frac{1}{N}\sum_{u \in \mathcal{P}}(m(y_u-\overline y))^2]^2 }-3
\] \[
\alpha(m*\mathcal{P}) = \frac{\frac{1}{N} \sum_{u \in \mathcal{P}}m^4\left( y_u - \overline{y} \right)^4}{ [\frac{1}{N}\sum_{u \in \mathcal{P}}m^2(y_u-\overline y)^2]^2 }-3
\] \(m\) is a constant, so it can be factored out of the sum and
cancelled. \[
\alpha(m*\mathcal{P}) = \frac{m^4\frac{1}{N} \sum_{u \in \mathcal{P}}\left( y_u - \overline{y} \right)^4}{ m^4[\frac{1}{N}\sum_{u \in \mathcal{P}}(y_u-\overline y)^2]^2 }-3
\] \[
\alpha(m*\mathcal{P}) = \frac{\frac{1}{N} \sum_{u \in \mathcal{P}}\left( y_u - \overline{y} \right)^4}{ [\frac{1}{N}\sum_{u \in \mathcal{P}}(y_u-\overline y)^2]^2 }-3 = \alpha(\mathcal{P})
\] Therefore, excess kurtosis is scale invariant.

\newpage

\hypertarget{question-1-c}{%
\subsubsection{Question 1 c)}\label{question-1-c}}

Since excess kurtosis is both location invariant (part a) and scale
invariant (part b), then by definition, it is location-scale invariant.

\newpage

\hypertarget{question-1-d}{%
\subsubsection{Question 1 d)}\label{question-1-d}}

\(\alpha(\mathcal{P})\) is scale invariant if
\(\alpha(\mathcal{P}^k) = \alpha(\mathcal{P})\). Population
\(\mathcal{P}\) is duplicated \(k-1\) times, and the new population size
is \(kN\). \[
\alpha(\mathcal{P}^k) = \frac{\frac{1}{kN} \sum_{u \in \mathcal{P}^k}\left( y_u -\overline{y} \right)^4}{ [\frac{1}{kN}\sum_{u \in \mathcal{P}^k}(y_u-\overline y)^2]^2 }-3
\] \[
\alpha(\mathcal{P}^k) = \frac{\frac{1}{kN} k\sum_{u \in \mathcal{P}}\left( y_u -\overline{y} \right)^4}{ [\frac{1}{kN}k\sum_{u \in \mathcal{P}}(y_u-\overline y)^2]^2 }-3
\] The summing operation is applied \(k\) times, but the population is
\(k\) times larger, so the factors cancel out. \[
\alpha(\mathcal{P}^k) = \frac{\frac{1}{N} \sum_{u \in \mathcal{P}}\left( y_u -\overline{y} \right)^4}{ [\frac{1}{N}\sum_{u \in \mathcal{P}}(y_u-\overline y)^2]^2 }-3 = \alpha(\mathcal{P})
\] Therefore, excess kurtosis is replication invariant.

\newpage

\hypertarget{question-1-e}{%
\subsubsection{Question 1 e)}\label{question-1-e}}

In this case, the equation for \(\sigma^4\) is changed. \[
\sigma_*^4 = [\frac{1}{N-1}\sum_{u \in \mathcal{P}}(y_u-\overline y)^2]^2
\]

For part b), the equation becomes: \[
\alpha(m*\mathcal{P}) = \frac{m^4\frac{1}{N} \sum_{u \in \mathcal{P}}\left( y_u - \overline{y} \right)^4}{ m^4[\frac{1}{N-1}\sum_{u \in \mathcal{P}}(y_u-\overline y)^2]^2 }-3
\] \[
\alpha(m*\mathcal{P}) = \frac{\frac{1}{N} \sum_{u \in \mathcal{P}}\left( y_u - \overline{y} \right)^4}{[\frac{1}{N-1}\sum_{u \in \mathcal{P}}(y_u-\overline y)^2]^2 }-3 = \alpha(\mathcal{P})
\] Since the multiplier \(m\) does not relate to the size of the
population \(N\), the attribute is still scale invariant (no change).
However, for part d), the equation becomes: \[
\alpha(\mathcal{P}^k) = \frac{\frac{k}{kN} \sum_{u \in \mathcal{P}}\left( y_u -\overline{y} \right)^4}{ [\frac{k}{kN-1}\sum_{u \in \mathcal{P}}(y_u-\overline y)^2]^2 }-3
\] \[
\alpha(\mathcal{P}^k) = \frac{\frac{1}{N} \sum_{u \in \mathcal{P}}\left( y_u -\overline{y} \right)^4}{ [\frac{1}{N-\frac{1}{k}}\sum_{u \in \mathcal{P}}(y_u-\overline y)^2]^2 }-3 \ne \alpha(\mathcal{P})
\] Therefore, \(\alpha(\mathcal{P})\) is no longer replication
invariant, but rather neither invariant nor equivariant.

\newpage

\hypertarget{question-1-f}{%
\subsubsection{Question 1 f)}\label{question-1-f}}

The sensitivity curve is defined as:
\(SC(y) = N(\alpha(\mathcal{P}^*) - \alpha(\mathcal{P}))\) \[
\alpha(\mathcal{P}^*) = \frac{\frac{1}{N} \sum_{u \in \mathcal{P}^*}\left( y_u -\overline{y} \right)^4}{ [\frac{1}{N}\sum_{u \in \mathcal{P}^*}(y_u-\overline y)^2]^2 }-3
\] \[
\alpha(\mathcal{P}) = \frac{\frac{1}{N-1} \sum_{u \in \mathcal{P}}\left( y_u -\overline{y} \right)^4}{ [\frac{1}{N-1}\sum_{u \in \mathcal{P}}(y_u-\overline y)^2]^2 }-3
\] Combined, the equation becomes: \[
SC(y) = N[(\frac{\frac{1}{N} \sum_{u \in \mathcal{P}^*}\left( y_u -\overline{y} \right)^4}{ [\frac{1}{N}\sum_{u \in \mathcal{P}^*}(y_u-\overline y)^2]^2 }-3) - (\frac{\frac{1}{N-1} \sum_{u \in \mathcal{P}}\left( y_u -\overline{y} \right)^4}{ [\frac{1}{N-1}\sum_{u \in \mathcal{P}}(y_u-\overline y)^2]^2 }-3)]
\] \[
SC(y) = N(\frac{\frac{1}{N} \sum_{u \in \mathcal{P}^*}\left( y_u -\overline{y} \right)^4}{ [\frac{1}{N}\sum_{u \in \mathcal{P}^*}(y_u-\overline y)^2]^2 } - \frac{\frac{1}{N-1} \sum_{u \in \mathcal{P}}\left( y_u -\overline{y} \right)^4}{ [\frac{1}{N-1}\sum_{u \in \mathcal{P}}(y_u-\overline y)^2]^2 })
\]

\newpage

\hypertarget{question-1-g}{%
\subsubsection{Question 1 g)}\label{question-1-g}}

\begin{Shaded}
\begin{Highlighting}[]
\FunctionTok{library}\NormalTok{(e1071)}
\FunctionTok{set.seed}\NormalTok{(}\DecValTok{341}\NormalTok{)}
\NormalTok{pop }\OtherTok{\textless{}{-}} \FunctionTok{rt}\NormalTok{(}\DecValTok{1000}\NormalTok{, }\DecValTok{10}\NormalTok{)}
\NormalTok{y }\OtherTok{\textless{}{-}} \FunctionTok{seq}\NormalTok{(}\SpecialCharTok{{-}}\DecValTok{10}\NormalTok{, }\DecValTok{10}\NormalTok{, }\AttributeTok{length.out=}\DecValTok{1001}\NormalTok{)}

\NormalTok{sc }\OtherTok{\textless{}{-}} \ControlFlowTok{function}\NormalTok{(y.pop, y, attr, ...) \{}
\NormalTok{  N }\OtherTok{\textless{}{-}} \FunctionTok{length}\NormalTok{(y.pop) }\SpecialCharTok{+} \DecValTok{1}
  \FunctionTok{sapply}\NormalTok{(y, }\ControlFlowTok{function}\NormalTok{(y.new) \{}
\NormalTok{    N }\SpecialCharTok{*}\NormalTok{ (}\FunctionTok{attr}\NormalTok{(}\FunctionTok{c}\NormalTok{(y.new, y.pop), ...) }\SpecialCharTok{{-}} \FunctionTok{attr}\NormalTok{(y.pop, ...))}
\NormalTok{  \})}
\NormalTok{\}}

\FunctionTok{plot}\NormalTok{(y, }\FunctionTok{sc}\NormalTok{(pop, y, kurtosis), }\AttributeTok{type=}\StringTok{"l"}\NormalTok{, }\AttributeTok{lwd=}\DecValTok{2}\NormalTok{,}
     \AttributeTok{main=}\StringTok{"Sensitivity Curve for Excess Kurtosis"}\NormalTok{,}
     \AttributeTok{ylab=}\StringTok{"sensitivity"}\NormalTok{, }\AttributeTok{xlab=}\StringTok{"y"}\NormalTok{)}
\end{Highlighting}
\end{Shaded}

\includegraphics{Assignment1_Markdown_files/figure-latex/unnamed-chunk-1-1.pdf}

Based on this plot, kurtosis is robust to additions because its
sensitivity curve is relative flat within {[}-5, 5{]}. However, it is
unbounded because the two ends go to infinity, so it can be easily
influenced by extreme outliers in the population.

\newpage

\hypertarget{question-1-h}{%
\subsubsection{Question 1 h)}\label{question-1-h}}

\begin{enumerate}
\def\labelenumi{\arabic{enumi}.}
\item
  \(\gamma(\mathcal P)\) measures how close values lie in relation to
  each other, not in relation to some constant like 0, so it should be
  location invariant. A cluster of points are just as spread out at
  \(\overline{y}=0\) as \(\overline{y}=b\).
\item
  \(\gamma(\mathcal P)\) should be scale equivariant. When a population
  of points are each multiplied by \(m\), the distance/spread between
  any two points are also multiplied by \(m\), so \(\gamma(\mathcal P)\)
  should reflect this change linearly.
\item
  \(\gamma(\mathcal P)\) should not be location-scale invariant nor
  equivariant, because location and scale have different ideal
  properties as seen in points 1 and 2. Choosing either would compromise
  location invariance or scale equivariance.
\end{enumerate}

\newpage

\hypertarget{question-2-a}{%
\subsubsection{Question 2 a)}\label{question-2-a}}

\begin{Shaded}
\begin{Highlighting}[]
\NormalTok{filename }\OtherTok{\textless{}{-}} \FunctionTok{paste}\NormalTok{(}\StringTok{"data/iris.csv"}\NormalTok{, }\AttributeTok{sep=}\StringTok{"/"}\NormalTok{)}
\NormalTok{iris }\OtherTok{\textless{}{-}} \FunctionTok{read.csv}\NormalTok{(filename, }\AttributeTok{header=}\ConstantTok{TRUE}\NormalTok{)}
\FunctionTok{nrow}\NormalTok{(iris)}
\end{Highlighting}
\end{Shaded}

\begin{verbatim}
## [1] 150
\end{verbatim}

\begin{Shaded}
\begin{Highlighting}[]
\FunctionTok{ncol}\NormalTok{(iris)}
\end{Highlighting}
\end{Shaded}

\begin{verbatim}
## [1] 5
\end{verbatim}

\newpage

\hypertarget{question-2-b}{%
\subsubsection{Question 2 b)}\label{question-2-b}}

\begin{Shaded}
\begin{Highlighting}[]
\FunctionTok{library}\NormalTok{(pander)}
\NormalTok{iris.freq }\OtherTok{\textless{}{-}} \FunctionTok{table}\NormalTok{(iris}\SpecialCharTok{$}\NormalTok{Species)}
\FunctionTok{pander}\NormalTok{(iris.freq, }\AttributeTok{type =} \StringTok{\textquotesingle{}grid\textquotesingle{}}\NormalTok{)}
\end{Highlighting}
\end{Shaded}

\begin{longtable}[]{@{}
  >{\centering\arraybackslash}p{(\columnwidth - 4\tabcolsep) * \real{0.19}}
  >{\centering\arraybackslash}p{(\columnwidth - 4\tabcolsep) * \real{0.25}}
  >{\centering\arraybackslash}p{(\columnwidth - 4\tabcolsep) * \real{0.25}}@{}}
\toprule
Iris-setosa & Iris-versicolor & Iris-virginica \\
\midrule
\endhead
50 & 50 & 50 \\
\bottomrule
\end{longtable}

\newpage

\hypertarget{question-2-c}{%
\subsubsection{Question 2 c)}\label{question-2-c}}

\begin{Shaded}
\begin{Highlighting}[]
\FunctionTok{print}\NormalTok{(}\FunctionTok{paste0}\NormalTok{(}\StringTok{"Species with the largest sepal widths: "}\NormalTok{, iris}\SpecialCharTok{$}\NormalTok{Species[iris}\SpecialCharTok{$}\NormalTok{SepalWidth }\SpecialCharTok{==} \FunctionTok{max}\NormalTok{(iris}\SpecialCharTok{$}\NormalTok{SepalWidth)][}\DecValTok{1}\NormalTok{]))}
\end{Highlighting}
\end{Shaded}

\begin{verbatim}
## [1] "Species with the largest sepal widths: Iris-setosa"
\end{verbatim}

\begin{Shaded}
\begin{Highlighting}[]
\FunctionTok{print}\NormalTok{(}\FunctionTok{paste0}\NormalTok{(}\StringTok{"Species with the smallest sepal widths: "}\NormalTok{, iris}\SpecialCharTok{$}\NormalTok{Species[iris}\SpecialCharTok{$}\NormalTok{SepalWidth }\SpecialCharTok{==} \FunctionTok{min}\NormalTok{(iris}\SpecialCharTok{$}\NormalTok{SepalWidth)][}\DecValTok{1}\NormalTok{]))}
\end{Highlighting}
\end{Shaded}

\begin{verbatim}
## [1] "Species with the smallest sepal widths: Iris-versicolor"
\end{verbatim}

\newpage

\hypertarget{question-2-d}{%
\subsubsection{Question 2 d)}\label{question-2-d}}

\begin{Shaded}
\begin{Highlighting}[]
\ControlFlowTok{for}\NormalTok{ (s }\ControlFlowTok{in} \FunctionTok{unique}\NormalTok{(iris}\SpecialCharTok{$}\NormalTok{Species))\{}
  \FunctionTok{print}\NormalTok{(}\FunctionTok{paste0}\NormalTok{(}\StringTok{"Average sepal length for "}\NormalTok{, s, }\StringTok{": "}\NormalTok{, }\FunctionTok{mean}\NormalTok{(iris}\SpecialCharTok{$}\NormalTok{SepalLength[iris}\SpecialCharTok{$}\NormalTok{Species }\SpecialCharTok{==}\NormalTok{ s])))}
\NormalTok{\}}
\end{Highlighting}
\end{Shaded}

\begin{verbatim}
## [1] "Average sepal length for Iris-setosa: 5.006"
## [1] "Average sepal length for Iris-versicolor: 5.936"
## [1] "Average sepal length for Iris-virginica: 6.588"
\end{verbatim}

\newpage

\hypertarget{question-2-e}{%
\subsubsection{Question 2 e)}\label{question-2-e}}

\begin{Shaded}
\begin{Highlighting}[]
\NormalTok{iris}\SpecialCharTok{$}\NormalTok{PetalRatio }\OtherTok{\textless{}{-}}\NormalTok{ iris}\SpecialCharTok{$}\NormalTok{PetalWidth}\SpecialCharTok{/}\NormalTok{iris}\SpecialCharTok{$}\NormalTok{PetalLength}

\CommentTok{\# Iris species with largest and smallest PetalRatios}
\FunctionTok{print}\NormalTok{(}\FunctionTok{paste0}\NormalTok{(}\StringTok{"Species with the largest petal ratio: "}\NormalTok{, iris}\SpecialCharTok{$}\NormalTok{Species[iris}\SpecialCharTok{$}\NormalTok{PetalRatio }\SpecialCharTok{==} \FunctionTok{max}\NormalTok{(iris}\SpecialCharTok{$}\NormalTok{PetalRatio)][}\DecValTok{1}\NormalTok{]))}
\end{Highlighting}
\end{Shaded}

\begin{verbatim}
## [1] "Species with the largest petal ratio: Iris-virginica"
\end{verbatim}

\begin{Shaded}
\begin{Highlighting}[]
\FunctionTok{print}\NormalTok{(}\FunctionTok{paste0}\NormalTok{(}\StringTok{"Species with the smallest petal ratio: "}\NormalTok{, iris}\SpecialCharTok{$}\NormalTok{Species[iris}\SpecialCharTok{$}\NormalTok{PetalRatio }\SpecialCharTok{==} \FunctionTok{min}\NormalTok{(iris}\SpecialCharTok{$}\NormalTok{PetalRatio)][}\DecValTok{1}\NormalTok{]))}
\end{Highlighting}
\end{Shaded}

\begin{verbatim}
## [1] "Species with the smallest petal ratio: Iris-setosa"
\end{verbatim}

\begin{Shaded}
\begin{Highlighting}[]
\CommentTok{\# Proportion of samples with PetalRatio \textgreater{} 0.3}
\ControlFlowTok{for}\NormalTok{ (s }\ControlFlowTok{in} \FunctionTok{unique}\NormalTok{(iris}\SpecialCharTok{$}\NormalTok{Species))\{}
  \FunctionTok{print}\NormalTok{(}\FunctionTok{paste0}\NormalTok{(}\StringTok{"Proportion of samples in "}\NormalTok{, s, }\StringTok{" with PetalRatio \textgreater{} 0.3: "}\NormalTok{, }\FunctionTok{mean}\NormalTok{(iris}\SpecialCharTok{$}\NormalTok{PetalRatio[iris}\SpecialCharTok{$}\NormalTok{Species }\SpecialCharTok{==}\NormalTok{ s] }\SpecialCharTok{\textgreater{}} \FloatTok{0.3}\NormalTok{)))}
\NormalTok{\}}
\end{Highlighting}
\end{Shaded}

\begin{verbatim}
## [1] "Proportion of samples in Iris-setosa with PetalRatio > 0.3: 0.04"
## [1] "Proportion of samples in Iris-versicolor with PetalRatio > 0.3: 0.66"
## [1] "Proportion of samples in Iris-virginica with PetalRatio > 0.3: 0.86"
\end{verbatim}

\newpage

\hypertarget{question-2-f}{%
\subsubsection{Question 2 f)}\label{question-2-f}}

\begin{Shaded}
\begin{Highlighting}[]
\CommentTok{\# Match each species name to a colour}
\NormalTok{species }\OtherTok{\textless{}{-}} \FunctionTok{unique}\NormalTok{(iris}\SpecialCharTok{$}\NormalTok{Species)}
\NormalTok{iris}\SpecialCharTok{$}\NormalTok{SpeciesNum }\OtherTok{\textless{}{-}} \FunctionTok{match}\NormalTok{(iris}\SpecialCharTok{$}\NormalTok{Species, species)}
\NormalTok{colours }\OtherTok{\textless{}{-}} \FunctionTok{c}\NormalTok{(}\StringTok{"black"}\NormalTok{, }\StringTok{"red"}\NormalTok{, }\StringTok{"green"}\NormalTok{)}

\CommentTok{\# Plot the points and add a legend}
\FunctionTok{plot}\NormalTok{(iris}\SpecialCharTok{$}\NormalTok{PetalLength, iris}\SpecialCharTok{$}\NormalTok{SepalLength, }\AttributeTok{pch=}\DecValTok{19}\NormalTok{,}
     \AttributeTok{col=}\NormalTok{colours[}\FunctionTok{as.numeric}\NormalTok{(iris}\SpecialCharTok{$}\NormalTok{SpeciesNum)], }
     \AttributeTok{xlab=}\StringTok{"petal length"}\NormalTok{, }\AttributeTok{ylab=}\StringTok{"sepal length"}\NormalTok{, }
     \AttributeTok{main=}\StringTok{"Petal Length vs. Sepal Length"}\NormalTok{)}
\FunctionTok{legend}\NormalTok{(}\DecValTok{1}\NormalTok{, }\FloatTok{7.75}\NormalTok{, }\AttributeTok{legend=}\NormalTok{species, }\AttributeTok{col=}\NormalTok{colours, }\AttributeTok{lty=}\DecValTok{1}\NormalTok{, }\AttributeTok{cex=}\FloatTok{0.8}\NormalTok{)}
\end{Highlighting}
\end{Shaded}

\includegraphics{Assignment1_Markdown_files/figure-latex/unnamed-chunk-7-1.pdf}

From the plot, it is clear that Iris-setosas can be distinguished from
Iris-versicolor and Iris-virginica by petal length alone, as there is a
clear separation between the groups.

In addition, there is a positive linear trend between petal length and
sepal length in Iris-versicolor and Iris-virginica.

\newpage

\hypertarget{question-2-g}{%
\subsubsection{Question 2 g)}\label{question-2-g}}

\begin{Shaded}
\begin{Highlighting}[]
\FunctionTok{par}\NormalTok{(}\AttributeTok{mfrow=}\FunctionTok{c}\NormalTok{(}\DecValTok{1}\NormalTok{,}\DecValTok{3}\NormalTok{), }\AttributeTok{mar=}\FunctionTok{c}\NormalTok{(}\DecValTok{2}\NormalTok{, }\DecValTok{2}\NormalTok{, }\DecValTok{2}\NormalTok{, }\DecValTok{1}\NormalTok{), }\AttributeTok{oma=}\FunctionTok{c}\NormalTok{(}\DecValTok{4}\NormalTok{, }\DecValTok{4}\NormalTok{, }\DecValTok{0}\NormalTok{, }\DecValTok{0}\NormalTok{))}
\ControlFlowTok{for}\NormalTok{ (s }\ControlFlowTok{in}\NormalTok{ species) \{}
  \FunctionTok{plot}\NormalTok{(}\FunctionTok{jitter}\NormalTok{(PetalWidth, }\AttributeTok{factor=}\FloatTok{0.5}\NormalTok{) }\SpecialCharTok{\textasciitilde{}} \FunctionTok{jitter}\NormalTok{(PetalLength, }\AttributeTok{factor=}\FloatTok{0.5}\NormalTok{), }
       \AttributeTok{data=}\NormalTok{iris[iris}\SpecialCharTok{$}\NormalTok{Species }\SpecialCharTok{==}\NormalTok{ s, ],}
       \AttributeTok{col=}\FunctionTok{adjustcolor}\NormalTok{(}\StringTok{"black"}\NormalTok{, }\FloatTok{0.3}\NormalTok{), }\AttributeTok{pch=}\DecValTok{19}\NormalTok{, }\AttributeTok{main=}\NormalTok{s)}
  \FunctionTok{points}\NormalTok{(}\FunctionTok{mean}\NormalTok{(PetalWidth) }\SpecialCharTok{\textasciitilde{}} \FunctionTok{mean}\NormalTok{(PetalLength), }\AttributeTok{pch=}\DecValTok{19}\NormalTok{,}
         \AttributeTok{data=}\NormalTok{iris[iris}\SpecialCharTok{$}\NormalTok{Species }\SpecialCharTok{==}\NormalTok{ s, ], }
         \AttributeTok{col=}\FunctionTok{adjustcolor}\NormalTok{(}\StringTok{"red"}\NormalTok{, }\FloatTok{0.5}\NormalTok{))}
\NormalTok{\}}
\FunctionTok{mtext}\NormalTok{(}\StringTok{\textquotesingle{}petal width\textquotesingle{}}\NormalTok{, }\AttributeTok{side=}\DecValTok{1}\NormalTok{, }\AttributeTok{outer=}\ConstantTok{TRUE}\NormalTok{, }\AttributeTok{line=}\DecValTok{2}\NormalTok{)}
\FunctionTok{mtext}\NormalTok{(}\StringTok{\textquotesingle{}petal length\textquotesingle{}}\NormalTok{, }\AttributeTok{side=}\DecValTok{2}\NormalTok{, }\AttributeTok{outer=}\ConstantTok{TRUE}\NormalTok{, }\AttributeTok{line=}\DecValTok{2}\NormalTok{)}
\end{Highlighting}
\end{Shaded}

\includegraphics{Assignment1_Markdown_files/figure-latex/unnamed-chunk-8-1.pdf}

The plots show that there is a slight positive linear relationship
between petal width and petal length, especially in the plot of
Iris-versicolor. The relationship might be clearer if the values were
not discrete/overlapping.

\newpage

\hypertarget{question-2-h}{%
\subsubsection{Question 2 h)}\label{question-2-h}}

\begin{Shaded}
\begin{Highlighting}[]
\CommentTok{\# Plot the base points}
\FunctionTok{par}\NormalTok{(}\AttributeTok{mfrow=}\FunctionTok{c}\NormalTok{(}\DecValTok{1}\NormalTok{,}\DecValTok{1}\NormalTok{))}
\FunctionTok{plot}\NormalTok{(iris}\SpecialCharTok{$}\NormalTok{SpeciesNum, iris}\SpecialCharTok{$}\NormalTok{SepalLength, }\AttributeTok{pch=}\DecValTok{19}\NormalTok{, }
     \AttributeTok{col=}\FunctionTok{adjustcolor}\NormalTok{(}\StringTok{"black"}\NormalTok{, }\FloatTok{0.3}\NormalTok{), }\AttributeTok{xaxt=}\StringTok{"n"}\NormalTok{, }
     \AttributeTok{xlab=}\StringTok{"species"}\NormalTok{, }\AttributeTok{ylab=}\StringTok{"sepal length"}\NormalTok{, }\AttributeTok{main=}\StringTok{"Sepal Length by Species"}\NormalTok{)}
\FunctionTok{axis}\NormalTok{(}\AttributeTok{side=}\DecValTok{1}\NormalTok{, }\AttributeTok{at=}\FunctionTok{c}\NormalTok{(}\FloatTok{1.0}\NormalTok{, }\FloatTok{2.0}\NormalTok{, }\FloatTok{3.0}\NormalTok{), }\AttributeTok{labels=}\NormalTok{species)}

\CommentTok{\# Calculate, plot, and connect the medians}
\NormalTok{medians }\OtherTok{\textless{}{-}} \FunctionTok{c}\NormalTok{(}\FunctionTok{median}\NormalTok{(iris}\SpecialCharTok{$}\NormalTok{SepalLength[iris}\SpecialCharTok{$}\NormalTok{SpeciesNum }\SpecialCharTok{==} \DecValTok{1}\NormalTok{]),}
             \FunctionTok{median}\NormalTok{(iris}\SpecialCharTok{$}\NormalTok{SepalLength[iris}\SpecialCharTok{$}\NormalTok{SpeciesNum }\SpecialCharTok{==} \DecValTok{2}\NormalTok{]),}
             \FunctionTok{median}\NormalTok{(iris}\SpecialCharTok{$}\NormalTok{SepalLength[iris}\SpecialCharTok{$}\NormalTok{SpeciesNum }\SpecialCharTok{==} \DecValTok{3}\NormalTok{]))}
\FunctionTok{points}\NormalTok{(}\DecValTok{1}\SpecialCharTok{:}\DecValTok{3}\NormalTok{, medians, }\AttributeTok{col=}\StringTok{"red"}\NormalTok{, }\AttributeTok{pch=}\DecValTok{19}\NormalTok{)}
\FunctionTok{lines}\NormalTok{(}\DecValTok{1}\SpecialCharTok{:}\DecValTok{3}\NormalTok{, medians, }\AttributeTok{col=}\StringTok{"red"}\NormalTok{, }\AttributeTok{pch=}\DecValTok{19}\NormalTok{)}
\end{Highlighting}
\end{Shaded}

\includegraphics{Assignment1_Markdown_files/figure-latex/unnamed-chunk-9-1.pdf}

\newpage

\hypertarget{question-2-i}{%
\subsubsection{Question 2 i)}\label{question-2-i}}

\begin{Shaded}
\begin{Highlighting}[]
\NormalTok{powerfun }\OtherTok{\textless{}{-}} \ControlFlowTok{function}\NormalTok{(x, alpha) \{}
  \ControlFlowTok{if}\NormalTok{(}\FunctionTok{sum}\NormalTok{(x }\SpecialCharTok{\textless{}=} \DecValTok{0}\NormalTok{) }\SpecialCharTok{\textgreater{}} \DecValTok{1}\NormalTok{) }\FunctionTok{stop}\NormalTok{(}\StringTok{"x must be positive"}\NormalTok{)}
  \ControlFlowTok{if}\NormalTok{ (alpha }\SpecialCharTok{==} \DecValTok{0}\NormalTok{)}
    \FunctionTok{log}\NormalTok{(x)}
  \ControlFlowTok{else} \ControlFlowTok{if}\NormalTok{ (alpha }\SpecialCharTok{\textgreater{}} \DecValTok{0}\NormalTok{) \{}
\NormalTok{    x}\SpecialCharTok{\^{}}\NormalTok{alpha}
\NormalTok{  \} }\ControlFlowTok{else} \SpecialCharTok{{-}}\NormalTok{x}\SpecialCharTok{\^{}}\NormalTok{alpha}
\NormalTok{\}}
\end{Highlighting}
\end{Shaded}

\begin{Shaded}
\begin{Highlighting}[]
\CommentTok{\# Part i: Histogram and scatterplots of SepalLength and PetalRatio}
\FunctionTok{par}\NormalTok{(}\AttributeTok{mfrow=}\FunctionTok{c}\NormalTok{(}\DecValTok{1}\NormalTok{,}\DecValTok{3}\NormalTok{), }\AttributeTok{mar=}\FunctionTok{c}\NormalTok{(}\DecValTok{2}\NormalTok{, }\DecValTok{2}\NormalTok{, }\DecValTok{2}\NormalTok{, }\FloatTok{0.2}\NormalTok{))}
\FunctionTok{hist}\NormalTok{(iris}\SpecialCharTok{$}\NormalTok{SepalLength, }\AttributeTok{breaks=}\DecValTok{15}\NormalTok{, }\AttributeTok{main=}\StringTok{"Sepal Length"}\NormalTok{)}
\FunctionTok{hist}\NormalTok{(iris}\SpecialCharTok{$}\NormalTok{PetalRatio, }\AttributeTok{breaks=}\DecValTok{15}\NormalTok{, }\AttributeTok{main=}\StringTok{"Petal Ratio"}\NormalTok{)}
\FunctionTok{plot}\NormalTok{(SepalLength }\SpecialCharTok{\textasciitilde{}}\NormalTok{ PetalRatio, }\AttributeTok{data=}\NormalTok{iris, }\AttributeTok{pch=}\DecValTok{19}\NormalTok{,}
     \AttributeTok{col=}\FunctionTok{adjustcolor}\NormalTok{(}\StringTok{"black"}\NormalTok{, }\FloatTok{0.5}\NormalTok{),}
     \AttributeTok{main=}\StringTok{"Sepal Length vs. Petal Ratio"}\NormalTok{)}
\end{Highlighting}
\end{Shaded}

\includegraphics{Assignment1_Markdown_files/figure-latex/unnamed-chunk-11-1.pdf}

\newpage

\begin{Shaded}
\begin{Highlighting}[]
\CommentTok{\# Part ii: Find alpha to make SepalLength symmetric}
\ControlFlowTok{for}\NormalTok{ (i }\ControlFlowTok{in} \FunctionTok{c}\NormalTok{(}\SpecialCharTok{{-}}\FloatTok{0.2}\NormalTok{, }\SpecialCharTok{{-}}\FloatTok{0.1}\NormalTok{, }\DecValTok{0}\NormalTok{, }\FloatTok{0.1}\NormalTok{, }\FloatTok{0.2}\NormalTok{)) \{}
  \FunctionTok{hist}\NormalTok{(}\FunctionTok{powerfun}\NormalTok{(iris}\SpecialCharTok{$}\NormalTok{SepalLength, i), }\AttributeTok{breaks=}\DecValTok{8}\NormalTok{)}
\NormalTok{\}}
\end{Highlighting}
\end{Shaded}

\begin{Shaded}
\begin{Highlighting}[]
\FunctionTok{hist}\NormalTok{(}\FunctionTok{powerfun}\NormalTok{(iris}\SpecialCharTok{$}\NormalTok{SepalLength, }\DecValTok{0}\NormalTok{), }\AttributeTok{breaks=}\DecValTok{8}\NormalTok{,}
     \AttributeTok{xlab=}\StringTok{"log(Sepal Length)"}\NormalTok{, }\AttributeTok{ylab=}\StringTok{"Frequency"}\NormalTok{, }\AttributeTok{main=}\StringTok{""}\NormalTok{)}
\end{Highlighting}
\end{Shaded}

\includegraphics{Assignment1_Markdown_files/figure-latex/unnamed-chunk-13-1.pdf}

After testing, the power 0 (log) makes sepal length most symmetric.

\newpage

\begin{Shaded}
\begin{Highlighting}[]
\CommentTok{\# Part iii: Find alpha to make PetalRatio symmetric}
\ControlFlowTok{for}\NormalTok{ (i }\ControlFlowTok{in} \FunctionTok{c}\NormalTok{(}\FloatTok{1.25}\NormalTok{, }\FloatTok{1.5}\NormalTok{, }\FloatTok{1.75}\NormalTok{, }\DecValTok{2}\NormalTok{, }\FloatTok{2.25}\NormalTok{)) \{}
  \FunctionTok{hist}\NormalTok{(}\FunctionTok{powerfun}\NormalTok{(iris}\SpecialCharTok{$}\NormalTok{PetalRatio, i), }\AttributeTok{breaks=}\DecValTok{30}\NormalTok{)}
\NormalTok{\}}
\end{Highlighting}
\end{Shaded}

\begin{Shaded}
\begin{Highlighting}[]
\FunctionTok{hist}\NormalTok{(}\FunctionTok{powerfun}\NormalTok{(iris}\SpecialCharTok{$}\NormalTok{PetalRatio, }\FloatTok{1.75}\NormalTok{), }\AttributeTok{breaks=}\DecValTok{15}\NormalTok{,}
     \AttributeTok{xlab=}\StringTok{"Petal Ratio \^{} 1.75"}\NormalTok{, }\AttributeTok{ylab=}\StringTok{"Frequency"}\NormalTok{, }\AttributeTok{main=}\StringTok{""}\NormalTok{)}
\end{Highlighting}
\end{Shaded}

\includegraphics{Assignment1_Markdown_files/figure-latex/unnamed-chunk-15-1.pdf}

A power of 1.75 makes the histogram of petal ratio most symmetric,
though it is harder to judge than sepal length due to the second peak on
the left.

\newpage

\begin{Shaded}
\begin{Highlighting}[]
\CommentTok{\# Part iv: Find the pair of alphas to make the scatterplot approximately linear}
\FunctionTok{plot}\NormalTok{(}\FunctionTok{powerfun}\NormalTok{(SepalLength, }\DecValTok{0}\NormalTok{) }\SpecialCharTok{\textasciitilde{}} \FunctionTok{powerfun}\NormalTok{(PetalRatio, }\FloatTok{1.75}\NormalTok{), }\AttributeTok{data=}\NormalTok{iris, }
     \AttributeTok{pch=}\DecValTok{19}\NormalTok{, }\AttributeTok{col=}\FunctionTok{adjustcolor}\NormalTok{(}\StringTok{"black"}\NormalTok{, }\FloatTok{0.5}\NormalTok{),}
     \AttributeTok{main=}\StringTok{"Power{-}Transformed Scatterplot"}\NormalTok{, }
     \AttributeTok{xlab=}\StringTok{"log(Sepal Length)"}\NormalTok{, }\AttributeTok{ylab=}\StringTok{"Petal Ratio \^{} 1.75"}\NormalTok{)}
\end{Highlighting}
\end{Shaded}

\includegraphics{Assignment1_Markdown_files/figure-latex/unnamed-chunk-16-1.pdf}

The new scatterplot uses the best alphas for sepal length (0) and petal
ratio (1.75), respectively, resulting in a more linear graph than the
one in part i).

\newpage

\hypertarget{question-3}{%
\subsubsection{Question 3}\label{question-3}}

\begin{Shaded}
\begin{Highlighting}[]
\NormalTok{drawBoxPlot }\OtherTok{\textless{}{-}} \ControlFlowTok{function}\NormalTok{(df) \{}
  \FunctionTok{plot}\NormalTok{(}\DecValTok{1}\NormalTok{, }\AttributeTok{type=}\StringTok{"n"}\NormalTok{, }\AttributeTok{xlab=}\StringTok{""}\NormalTok{, }\AttributeTok{ylab=}\StringTok{""}\NormalTok{, }\AttributeTok{xaxt=}\StringTok{"n"}\NormalTok{,}
       \AttributeTok{xlim=}\FunctionTok{c}\NormalTok{(}\FloatTok{0.5}\NormalTok{, }\FunctionTok{ncol}\NormalTok{(df)}\SpecialCharTok{+}\FloatTok{0.5}\NormalTok{), }\AttributeTok{ylim=}\FunctionTok{c}\NormalTok{(}\FunctionTok{min}\NormalTok{(df), }\FunctionTok{max}\NormalTok{(df)))}
  
\NormalTok{  w1 }\OtherTok{=} \FloatTok{0.25}
\NormalTok{  w2 }\OtherTok{=} \FloatTok{0.4}
  \ControlFlowTok{for}\NormalTok{ (i }\ControlFlowTok{in} \DecValTok{1}\SpecialCharTok{:}\FunctionTok{ncol}\NormalTok{(df))\{}
\NormalTok{    bstats }\OtherTok{=} \FunctionTok{boxplot.stats}\NormalTok{(df[,i])}
\NormalTok{    stats }\OtherTok{=}\NormalTok{ bstats}\SpecialCharTok{$}\NormalTok{stats}
    
    \FunctionTok{segments}\NormalTok{(i, stats[}\DecValTok{1}\NormalTok{], i, stats[}\DecValTok{5}\NormalTok{], }\AttributeTok{lty=}\DecValTok{2}\NormalTok{)}
    \FunctionTok{segments}\NormalTok{(i}\SpecialCharTok{{-}}\NormalTok{w1, stats[}\DecValTok{1}\NormalTok{], i}\SpecialCharTok{+}\NormalTok{w1, stats[}\DecValTok{1}\NormalTok{], }\AttributeTok{lty=}\DecValTok{1}\NormalTok{)}
    \FunctionTok{segments}\NormalTok{(i}\SpecialCharTok{{-}}\NormalTok{w1, stats[}\DecValTok{5}\NormalTok{], i}\SpecialCharTok{+}\NormalTok{w1, stats[}\DecValTok{5}\NormalTok{], }\AttributeTok{lty=}\DecValTok{1}\NormalTok{)}
    \FunctionTok{rect}\NormalTok{(i}\SpecialCharTok{{-}}\NormalTok{w2, stats[}\DecValTok{2}\NormalTok{], i}\SpecialCharTok{+}\NormalTok{w2, stats[}\DecValTok{4}\NormalTok{], }\AttributeTok{col=}\StringTok{"gray"}\NormalTok{)}
    \FunctionTok{segments}\NormalTok{(i}\SpecialCharTok{{-}}\NormalTok{w2, stats[}\DecValTok{3}\NormalTok{], i}\SpecialCharTok{+}\NormalTok{w2, stats[}\DecValTok{3}\NormalTok{], }\AttributeTok{lty=}\DecValTok{1}\NormalTok{, }\AttributeTok{lwd=}\DecValTok{3}\NormalTok{)}

    \FunctionTok{points}\NormalTok{(}\AttributeTok{x=}\FunctionTok{rep}\NormalTok{(i, }\FunctionTok{length}\NormalTok{(bstats}\SpecialCharTok{$}\NormalTok{out)), }\AttributeTok{y=}\NormalTok{bstats}\SpecialCharTok{$}\NormalTok{out)}
\NormalTok{  \}}
  \FunctionTok{axis}\NormalTok{(}\AttributeTok{side=}\DecValTok{1}\NormalTok{, }\AttributeTok{at=}\DecValTok{1}\SpecialCharTok{:}\FunctionTok{ncol}\NormalTok{(df), }\AttributeTok{labels=}\FunctionTok{names}\NormalTok{(df), }\AttributeTok{cex.axis=}\FloatTok{0.8}\NormalTok{)}
\NormalTok{\}}

\FunctionTok{drawBoxPlot}\NormalTok{(iris[}\FunctionTok{c}\NormalTok{(}\StringTok{"SepalLength"}\NormalTok{, }\StringTok{"SepalWidth"}\NormalTok{, }\StringTok{"PetalLength"}\NormalTok{, }\StringTok{"PetalWidth"}\NormalTok{)])}
\end{Highlighting}
\end{Shaded}

\includegraphics{Assignment1_Markdown_files/figure-latex/unnamed-chunk-17-1.pdf}

\newpage

\hypertarget{question-4}{%
\subsubsection{Question 4}\label{question-4}}

\end{document}
